\documentclass{article}
\usepackage[utf8]{inputenc, }
\usepackage{graphicx}

\graphicspath{{images/}}

\title{EE230 Homework-1}
\author{İsmail Enes Bülbül, Abdullah Emir Göğüsdere}
\date{March 2022}

\begin{document}
\maketitle
    \section*{1.}
    \subsection*{a)}
    \begin{math}
        \Omega = \{ x | x \in N, x > 0 \} 
    \end{math}
    \\
    The sample space consists of natural numbers greater than 0.
    The number represents number of total flipping before someone wins.
    For example, 2 represents that Ayşe get tail in the first trial,
    and Bora get head in his first trial.
    Any number represents that Ayşe wins if the remainder from the
    division of the number by 3 is 1, and represents Bora wins if
    the remainder is 2, and represents Ceyda wins otherwise.
    \subsection*{b)}
    \begin{math}
        A = \{ x | x \in \Omega, x  \% 3 = 1 \} \\
        B = \{ x | x \in \Omega, x  \% 3 = 2 \} \\
        A\cup B = \{ x | x \in \Omega, x  \% 3 \neq 0 \} \\
        (A\cup B)^c = \{ x | x \in \Omega, x  \% 3 = 0 \} \\
    \end{math}
    \subsection*{c)}
    Let the events \(E_k\) be defined as follows:\\
    \(E_k\): the event for the case that first k-1 trial is tail and kth trial is head.\\
    Then we can write events A and B in terms of \(E_k\) as follows:\\
    \begin{math}
        A = E_1 \cup E_4 \cup E_7 \cup ...\\
        B = E_2 \cup E_5 \cup E_8 \cup ...\\
    \end{math}
    Since these events are disjoint, we can calculate the probabilities as follows:
    \begin{math}
        P(A) = P(E_1) + P(E_4) + P(E_7) + ...\\
        P(B) = P(E_2) + P(E_5) + P(E_8) + ...\\
    \end{math}
    The probability of \(E_k\) can be calculated as:\\
    \begin{math}
        P(E_k) = (1-p)^{k-1}*p
    \end{math}
    
\end{document}
